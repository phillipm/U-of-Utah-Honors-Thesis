\chapter{INTRODUCTION}\label{ch:intro}
% TODO:
INTRO

\begin{SaveVerbatim}{const}
const :: a -> b -> a
\end{SaveVerbatim}

\begin{SaveVerbatim}{constProof}
const a b = a
\end{SaveVerbatim}

\section{What is the Curry-Howard Isomorphism?}
The Curry-Howard Isomorphism states that there is a direct correspondence between intuitionistic logic and typed lambda calculus. More specifically a type declaration given in a programming language corresponds to a logical proposition and an implementation of that type declaration is a proof of that proposition. Figure \ref{fig:const} shows the Haskell type signature for $const$ and the corresponding logical proposition via the Curry-Howard Isomorphism. Since intuitionistic logic is a special subset of classical logic, providing a well-typed implementation to the function's type signature is the same as giving a logical proof. The proof of $const$ is \BUseVerbatim{constProof}. The example in figure \ref{fig:const} is somewhat uninteresting since it deals with first-order logic. However, as will be shown, the Curry-Howard Isomorphism holds for higher levels of logic as well.

\begin{figure}
  \begin{subfigure}[b]{.5\linewidth}
    \BUseVerbatim{const}
    \vspace{.2in}
    \caption{Haskell type signature}
  \end{subfigure}
  \begin{subfigure}[b]{.5\linewidth}
    \centering
    \begin{equation*}
    \frac{\Gamma, \alpha \vdash \beta}{\Gamma \vdash \alpha \rightarrow \beta \rightarrow \alpha}
    \end{equation*}
    \caption{Intuitionistic logical proposition}
  \end{subfigure}
  \caption{Corresponding $const$ type signature and logical proposition}
  \label{fig:const}
\end{figure}

\section{Why is this interesting?}
One might ask how useful the C-H is? While it isn't directly applicable to the average programmer, it is heavily used in research. Programming Language researchers are coming up with crazy new programming languages that make use of C-H. For example, Cayenne and Omega, two relatively new languages use dependent types, which correspond to higher-order logic. The C-H enables Mathematicians and Computer Scientists to benefit from each other's discoveries. Computer scientists have traditionally researched reductions in lambda calculus, while mathematicians have worked on normalization in proof theory. Thus, a discovery in one domain directly translates into the other domain \cite{CHnotes}.

% TODO:
CLOSE
%My interest in C-H started when I wanted to create a system to generate functions given a type signature and example input-output pairs. The problem is too under-determined to be plausible, but using more advanced type systems and leveraging C-H might mitigate that.
