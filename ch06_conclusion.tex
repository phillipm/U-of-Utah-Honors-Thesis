\chapter{CONCLUSION}
The Curry-Howard Isomorphism relates intiutionistic logic with programming language type systems.
We've explored the ideas behind intiutionistic logic and how it differs from the notions of classical logic.
With System F, we got a feel for type systems that allow for quantification over predicates, or types.
In addition to polymorphism, quantification over predicates enabled us to derive certain theorems about our programs for free.
We concluded with a study of dependent types, which move much of the program's computation into the type level, yielding proof-carrying code.
Throughout the survey special attention was paid to demonstrating how these ideas have found their way into modern programming language implementations such as Haskell.
The Curry-Howard Isomorphism has yielded interesting results in both programming language theory and implementation. We must wonder, what else does it have to offer?
