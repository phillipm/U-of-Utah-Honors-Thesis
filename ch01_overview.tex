%
% $Id: ch01_overview
%
\chapter{Overview}\label{ch:overview}

\begin{figure}
\centering
\begin{tabular}{l l l}
\epsfxsize=1.8in\epsffile{Figdir/Nausicaa.eps} &
\epsfxsize=1.8in\epsffile{Figdir/Nausicaa.eps} &
\epsfxsize=1.8in\epsffile{Figdir/Nausicaa.eps} \\
\end{tabular}
\caption{M-rep based models.[replace with CAD-style model with displacements,
when we've got one]}\label{fig:teaser}
\end{figure}

\begin{quote}
\small
``If you'd like pithy remarks and cute quotes to begin each chapter, they can
be inserted like this"\\
---The Editor
\end{quote}

In the little more than 30 years since computer-based 3D geometric modeling
has been practiced, blah blah blah

\section{Conventional geometric modeling primitives} \label{sec:motivation}

\section{Thesis statement, claims and contributions}

Introduce your own approach to the problem in relation to one described, and then
give your thesis statement.

\begin{Thesis}
It is both simple and useful to provide students with a \LaTeX\ template
for their formal thesis proposals.  This results in far fewer
questions from students uncertain as to what belongs in their proposals.  Such a template
will be of benefit to students in Computer Science/Applied Computing as well as to any
others willing and able to use \LaTeX\ for their work.
\end{Thesis}

A formal thesis statement should be a \emph{falsifiable} statement about the goal
you will attempt to achieve with your research project.  For a purely scientific
project, this is the hypothesis you are testing with your research.  For an applied
programming project, it is usually a statement about the feasibility and correctness
of your approach and the advantages it has over other approaches.  For a survey, study,
or library comp, it is usually a statement regarding the need or usefulness of such
a study, its intended audience, and so on.

During my research, I have achieved the following results:

\begin{table*}
\centering
\begin{tabular}{l|c|c|c|c}
\emph{Primitives}& \emph{Global geometry}& \emph{Tolerance} & \emph{Multiscale}
& \emph{Shape statistics}
\\
\hline
B-rep models            & No        & No    & No & Yes\\
\ \ subdivision models  & No        & No    & Yes & No \\
\ \ sph.\ harmonic models & Only!    & No    & Somewhat & Yes \\
\hline
CSG models              & Yes       & No    & No & No \\
\hline
Implicit models         & No        & No    & Somewhat & No\\
\ \ convolution surfaces & Yes      & No    & Somewhat & No\\
\hline
Volume models           & No        & No    & No & No\\
\hline
Medial models           & Yes       & No    & Some & Yes \\
\hline
IBR models              & No        & No    & No & No \\
\hline
M-rep models            & Yes       & Yes   & Yes & Yes
\end{tabular}
\caption{\label{tb:Table1} Modeling primitives and their features.}
\end{table*}

\section{Thesis outline}
