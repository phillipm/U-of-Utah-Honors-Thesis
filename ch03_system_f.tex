\chapter{SYSTEM F} \label{ch:systemf}
System F is a second-order typed lambda calculus that allows universal quantification over types. The Curry-Howard Isomorphism extends to System F, providing a direct correspondence to second order intuitionistic logic. It was discovered independently by the Logician Jean-Yves Girard and Computer Scientist John C. Reynolds. Girard developed System F as a proof notation for second-order logic, while Reynolds invented System F when trying to build a type system for a polymorphic programming language. \cite{CHnotes}

\section{A Quick Intuition on System F}
The practical idea of System F is to add polymorphism to typed lambda calculus. For instance if we want to define the untyped function $composition = \lambda f. \lambda g. \lambda x. f\ (g\ x) $ in $\lambda_{\rightarrow} $ (simply typed lambda calculus), we would potentially need to define an infinite number of functions in order to satisfy all possible types:
\begin{spacing}{1}
\begin{equation*}
  \begin{aligned}
    & compositionInt = \lambda f:Int \rightarrow Int.\ \lambda g:Int \rightarrow Int.\ \lambda x:Int.\ f\ (g\ x) \\
    & compositionBool = \lambda f:Bool \rightarrow Bool.\ \lambda g:Bool \rightarrow Bool.\ \lambda x:Bool.\ f\ (g\ x)\\
    & \ldots
  \end{aligned}
\end{equation*}
\end{spacing}
\vspace{.1in}


To circumvent this, we add universal quantification ($\forall $) over types to the calculi and call it $\lambda_{2} $. The composition function now looks like this:
\begin{spacing}{1}
\begin{equation*}
  composition = \lambda X.\ \lambda f:X \rightarrow X.\ \lambda g:X \rightarrow X.\ \lambda x:X.\ f\ (g\ x)
\end{equation*}
\end{spacing}
\vspace{.1in}
The idea being that the type is abstracted out of the term, allowing for type variables to later be applied to it: $composition [Int] \rightarrow [X \mapsto Int] composition $. More formally, type abstractions and applications are added to the syntax and reduction rules of $\lambda_{\rightarrow} $ (see pg. 343 of \cite{tapl} for the rules).

Adding universal quantification over types means that we're now working with second order logic, which we will explore now.

\section{Fun with Quantifiers}
First-order logic only uses variables that range over individuals (elements of the domain of discourse) \cite{secondOrderWiki}. For instance, $\forall x \in S\ .\ (x+1 > 12)\ where\ S = \{1,2, \ldots, 10\}$

In second-order logic variables can also range over sets of individuals. As in it allows for quantification over not only individuals (terms in System F), but also over predicates (types in System F). For example: $\forall x \in Odds(S)\ .\ \exists y \in Evens(S)\ .\ (x+y = 11)\ where\ S = \{1,2, \ldots, 10\}$.

We must remember that we are still playing with intuitionistic logic, so it is worth exploring what quantifiers really mean in this context.  %Page 192 of 
\cite{CHnotes} does a good job of explaining the difference between quantification in classical and intuitionistic logics:
\begin{quote}
``The intended meaning of $\forall p \varphi(p)$ is that $\varphi(p) $ holds for all possible meanings of $p $. The meaning of $\exists p \varphi(p)$ is that $\varphi(p) $ holds for some meaning of $p $. Classically, there are just two possible such meanings: the two truth values. Thus, the statement $\forall p \varphi(p) $ is classically equivalent to $\varphi(\top) \wedge \varphi(\bot)$, and $\exists p \varphi(p) $ is equivalent to $\varphi(\top) \vee \varphi(\bot)$. \ldots In the intuitionistic logic, there is no finite set of truth-values, and the propositional quantifiers should be regarded as ranging over some infinite space of predicates.''
\end{quote}

\section{Correspondence in the Girard-Reynolds Isomorphism}
According to Wadler in \cite{Wadlergirard}, in addition to System F, Girard proved a Representation Theorem: every function on natural numbers that can be proved total in second-order predicate calculus can be represented in System F. While Reynolds proved an Abstraction Theorem: every term in System F satisfies a suitable notion of logical relation.

Girard's Representation Theorem results in a projection from second order logic into System F, while Reynolds' Abstraction Theorem results in an embedding of System F into second order logic.  Quoting Wadler loosely from \cite{Wadlergirard}: Second order logic is larger than System F's image under the Curry-Howard isomorphism. Second order logic contains quantifiers over individuals, types, and predicates, while System F only allows for quantification over types. Girard's projection from second order logic onto System F is similar to the Curry-Howard isomorphism, in that it takes propositions to types and proofs to terms, but differs in that it erases information about individuals. This mapping preserves reductions, hence it is no mere surjection but a true epimorphism.

Thus the Reynolds embedding followed by the Girard projection is the identity. While Girard's projection followed by Reynolds' embedding is not since the projection discards all information about individuals.

\section{System F in Use}
System F enables us to reason about programs with polymorphism. This power comes with a downside, type inference becomes undecidable. To mitigate this many modern languages use weaker forms of polymorphism, such as \emph{Rank-2 Polymorphism}, in which type inference is decidable. In addition to formalizing polymorphism, System F has yielded interesting results in language security (\cite{infoFlow}) and lead to the notion of free theorems.
