\begin{abstract}
The Curry-Howard Isomorphism is the correspondence between the intuitionistic fragment of classical logic and simply typed lambda calculus ($\lambda_\rightarrow$). It states that the type signatures of functions in $\lambda_\rightarrow$ correspond to logical propositions, and function bodies are equivalent to proofs of those propositions. In this survey we will explore the implications of this isomorphism and how it scales to higher order logic and type systems, such as System F and Dependent Types. Furthermore, we will look at how the Curry-Howard Isomorphism is being put to use in modern programming languages and academic research.
\end{abstract}
